%% This is the real main file for Tutorial of the ISC in FGrankfurt, 2020

\usepackage{pgfplots}
\usepackage{adjustbox}
\usepackage[toc,acronym]{glossaries}
\usepackage[customcolors,shade]{hf-tikz}
\definecolor{webgreen}{rgb}{0,.5,0}
\definecolor{webbrown}{rgb}{.6,0,0}
\definecolor{webyellow}{rgb}{0.98,0.92,0.73}
\definecolor{webgray}{rgb}{.753,.753,.753}
\definecolor{webblue}{rgb}{0,0,.8}
\definecolor{webgreen}{rgb}{0, 0.5, 0} % less intense green
\definecolor{webred}{rgb}{0.5, 0, 0}   % less intense red


%\renewcommand{\dbltopfraction}{0.9}	% fit big float above 2-col. text
\renewcommand{\textfraction}{0.05}	% allow minimal text w. figs
%   Parameters for FLOAT pages (not text pages):
\renewcommand{\floatpagefraction}{0.75}	% require fuller float pages
% N.B.: floatpagefraction MUST be less than topfraction !!
%\renewcommand{\dblfloatpagefraction}{0.7}	% require fuller float pages


\makeglossaries

\newacronym{AI}{AI}{Artificial Intelligence}
\newacronym{ISA}{ISA}{Instruction Set Architecture}
\newacronym{I/O}{I/O}{Input/Output}
\newacronym{MC}{MC}{Multi-Core and/or Many-Core}
\newacronym{MLP}{MLP}{Memory Level Parallelism}
\newacronym{OoO}{OoO}{Out-of-Order}
\newacronym{OS}{OS}{operating system}
\newacronym{PD}{PD}{Propagation Delay}
\newacronym{PU}{PU}{Processing Unit}
\newacronym{SPA}{SPA}{Single Processor Approach}
\newacronym{HPL}{HPL}{High Performance Linpack}
\newacronym{HPCG}{HPCG}{High Performance Conjugate Gradients}

\newcommand*{\boxcolor}{orange}
\makeatletter
\renewcommand{\boxed}[1]{\textcolor{\boxcolor}{%
		\tikz[baseline={([yshift=-1ex]current bounding box.center)}] \node [rectangle, minimum width=1ex,rounded corners,draw] {\normalcolor\m@th$\displaystyle#1$};}}
\makeatother
\begin{document}
	\MEfrontmatter
	\MEmainmatter
%%% Tutorial A, beginner's level
%\input{src/OverviewA.tex}

%\subsection{Detailed outline of the tutorial '\textit{Making the basic terms and ideas of HPC clear}' (with time slots)}
%\input{src/LessonA1.tex}
%\input{src/LessonA2.tex}
%\input{src/LessonA3.tex}
%\input{src/LessonA4.tex}
%
%%%% Tutorial B, advanced level	    
%
%%\subsection{Detailed outline of the tutorial '\textit{Using HPC for 	supercomputing}' (with time slots)}
\input{src/OverviewB.tex}
%\input{src/LessonB1.tex}
%\input{src/LessonB2.tex}
\input{src/LessonB3.tex}
%\input{src/LessonB4.tex}

%\input{src/WrapUp.tex}
 \bibliographystyle{IEEEtran}
\bibliography{../../CommonBibliography,%
	../../CommonPrivateBibliography%
}
	\MEbackmatter
\end{document}

%egregious = extremely bad in a way that is very noticeable
%	\input{src/Overview}
%	\part{Basic tutorial}
%  	\MEchapter[Single Processor Approach]{SPA: Around single-processor performance}
%    	\MEsection[Single-Processor Performance]{The Single-Processor Approach}
%    	\MEsection[Limitations of SPA]{Limitations of single-processor performance (Moore's Law)}
%    	\MEsection [Measure performance]{How to measure single-processor performance}
% 		\MEsection[Sequential and parallel] {What is sequential, parallel and parallelized sequential}
%		\MEsection[What is parallel]{What is parallel and concurrent and parallelized sequential performance special in}
%		
% 	\MEchapter[Limitations of parallelization]{Limitations of parallelized sequential performance (Amdahl's Law)}
%		\MEsection[Limitations]{Limitations of parallelized performance: Amdahl's Law}
%      	\MEsection[Amdahl's Law]{Amdahl's definition and model}	
%%		\input{src/Model}
%		\MEsection[The effective parallelization]{The role of the effective parallelization }
%		\MEsection[The "dark performance"]{The "dark performance" or the "supercomputer efficiency"}
%   		\MEsection[Amdahl's abusing]{Abusing and misinterpreteting Amdahl's Law}	
%
%		
%  	\MEchapter[Parallel performance]{The performance of parallelization}
%		\MEsection[What is wrong with SPA]{What is wrong with parallelization in SPA}
%		\MEsection[Many-core parallelization]{Many-core parallelization}
%		\MEsection[What kinds of performances exist]{What kinds of performances exist}
%		\MEsection[SW parallelization]{SW parallelization}
%		\MEsection[Parallelization in the cloud]{Parallelization in the cloud}
%	
%  	\MEchapter[Modern computing I]{Analogy with the sciences: counter-intuitive, shocking but true}
%	  	\MEsection[Relativity and computing]{Relativity and computing performance}
%	  	\MEsection[The measurement paradox]{The measurement paradox}
%
%	\part{Performance of HPC}
%  	\MEchapter[High performance] {How to achieve high performance}
%  		\MEsection[Optimization]{Single-thread vs many-processor optimization}
% 		\MEsection[Supercomputing performance]{The performance of supercomputing}
%		\MEsection[The contributions]{The contribution to effective parallelization}
%		\MEsection[The limiting factors]{The limiting factors}
%		
%  	\MEchapter[Benchmark performance]{What actually the benchmark programs measure}
%		\MEsection[Performance of supercomputing]{The performance limitations of supercomputing}
%		\MEsection[Benchmarking]{What actually the benchmark programs measure}
%		\MEsection[Half precision]{Reducing Operand length}
%		\MEsection[Workflow]{The role of workflow mode}
%		\MEsection[Why some supercomputers failed]{Why some supercomputers failed (and why more will fail)}
%		
%  	\MEchapter[Modern computing II]{Analogy with the sciences: counter-intuitive, shocking but true}
%		\MEsection[Collapse]{Gravitational and communicational collapse}
%		\MEsection[The quantal nature]{The quantal nature of (computing) time}
%		\MEsection[Interaction and communication]{Collective behavior}
%
%	\MEchapter[The exa-scale race]{What required to enter the exa-scale race}
%		
%		\MEsection[The "dark performance"]{How to increase the "dark performance"}
%  		\MEsection[The Explicitly Many-Processor Approach]{EMPA: a posible way out}
%	
%\section{Detailed description of the tutorial content (3 pages maximum)}
